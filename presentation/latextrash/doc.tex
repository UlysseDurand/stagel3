\newcommand{\letitle}{FOUINE}
\newcommand{\leauthor}{Benjamin Duhamel \& Ulysse Durand}

\documentclass{beamer}

\usepackage[utf8]{inputenc}
\usepackage{amssymb}
\usepackage{amsmath}
\usepackage{xcolor}
\usepackage{enumitem}
\usepackage{bbold}
\usepackage[xcolor,leftbars]{changebar}
\setlength{\parindent}{0pt}
\setcounter{secnumdepth}{1}
\newenvironment{myindentpar}
 {\begin{list}{}
         {\setlength{\leftmargin}{1em}}
         \item[]
 }
 { \end{list}}

\definecolor{DarkBlue}{RGB}{0,16,80}
\newcommand{\norm}[1]{\lvert #1 \rvert}
\newsavebox{\mybox}
\newlength{\mydepth}
\newlength{\myheight}
\newenvironment{answer}
{\par\begin{lrbox}{\mybox}\quad\begin{minipage}{\linewidth}\color{black}\setlength{\parskip}{10pt plus 1pt minus 1pt}\vspace*{-.7\baselineskip}}
{\end{minipage}\end{lrbox}
\settodepth{\mydepth}{\usebox{\mybox}}
\settoheight{\myheight}{\usebox{\mybox}}
\addtolength{\myheight}{\mydepth}
\noindent\makebox[0pt]{
  \color{gray}\hspace{-0pt}\rule[-\mydepth]{1pt}{\myheight}}
  \usebox{\mybox}
  }

\usepackage{listings}
\lstset
{
    language=Caml,
    basicstyle=\footnotesize,
    numbers=left,
    stepnumber=1,
    showstringspaces=false,
    tabsize=1,
    breaklines=true,
    breakatwhitespace=false,
}

\usepackage{hyperref}
\setlength{\parskip}{0.15cm}
\hypersetup{
    colorlinks=truem,
    linkcolor=black,
    filecolor=red,
    urlcolor=blue
}

\urlstyle{same}

\everymath{\displaystyle}
\title{\letitle}
\author{\leauthor}
\institute{}
\date{}
\begin{document}

\maketitle
\begin{frame}
La fouine est un interpréteur \texttt{ OCaml } qui a pour but de faire fouiner
ses créateurs dans l'implémentation d'un langage fonctionnel.

\end{frame}
\begin{frame}
\frametitle{Exemple de code}


\texttt{ let rec f x n = if n = 0 then x else f (match x with |(y,z) -> (z, y+z)) (n-1) in f (0,1) 10 }

est un code \texttt{ OCaml } qui permettrait de calculer le dixième terme de la suite de fibonacci

C'est aussi un code \texttt{ fouine }

\end{frame}
\begin{frame}
\frametitle{Les expressions}


On traduit le code en un arbre d'expression

Exemple avec \texttt{ if true then (3+4)*7 else false }

\end{frame}
\begin{frame}
\frametitle{Implémentation des références}


On a une mémoire qui est un long array, l'adressage mémoire est naïf.

\end{frame}
\begin{frame}
\frametitle{Implémentation des fonctions}


Pour une fonction représentée par \texttt{ VFun (x, e, env) },
appliquée à une expression \texttt{ e2 }, on évalue \texttt{ e2 }
en ajoutant à \texttt{ env } que \texttt{ x } vaut \texttt{ e }

\end{frame}
\begin{frame}
\frametitle{Pour les fonctions récursive}

La représentation est différente, \texttt{ VFunRec (f, e, env) }

On fait de même mais en "déroulant" l'environnement env, c'est à dire
en remettant ce même \texttt{ env } dans les environnements des \texttt{ VFunRec } de \texttt{ env }.

Exemple avec

\texttt{ let rec f n = if n = 0 then 1 else n * (f (n-1)) in f 10 }


\end{frame}
\begin{frame}
\frametitle{Implémentation des boucles (While et For)}


Plutôt simple
\end{frame}
\begin{frame}
\frametitle{Implémentation des couples et listes (avec leur matching)}


Plutôt compliqué
\end{frame}
\end{document}
