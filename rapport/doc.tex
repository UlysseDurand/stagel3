\newcommand{\letitle}{Stage de L3, formalisation de Game Semantics avec Coq}
\newcommand{\leauthor}{Ulysse Durand\\\\\\dirigé par \\\\Pierre Clairambault\\ et \\Etienne Miquey}

\documentclass[a4paper,12ptCOUCOU
]{article}
\usepackage{tikzit}

\usepackage[utf8]{inputenc}
\usepackage{amssymb}
\usepackage[style=numeric]{biblatex}
\usepackage{amsmath}
\usepackage{xcolor}
\usepackage{enumitem}
\usepackage{tikz}
\usepackage{tikzit}
\usepackage{bbold}
\usepackage[xcolor,leftbars]{changebar}
\setlength{\parindent}{0pt}
\setcounter{secnumdepth}{1}
\newenvironment{myindentpar}
 {\begin{list}{}
         {\setlength{\leftmargin}{1em}}
         \item[]
 }
 { \end{list}}




\definecolor{DarkBlue}{RGB}{0,16,80}
\newcommand{\norm}[1]{\lvert #1 \rvert}
\newsavebox{\mybox}
\newlength{\mydepth}
\newlength{\myheight}
\newenvironment{answer}
{\par\begin{lrbox}{\mybox}\quad\begin{minipage}{\linewidth}\color{black}\setlength{\parskip}{10pt plus 1pt minus 1pt}\vspace*{-.7\baselineskip}}
{\end{minipage}\end{lrbox}
\settodepth{\mydepth}{\usebox{\mybox}}
\settoheight{\myheight}{\usebox{\mybox}}
\addtolength{\myheight}{\mydepth}
\noindent\makebox[0pt]{
  \color{gray}\hspace{-0pt}\rule[-\mydepth]{1pt}{\myheight}}
\usebox{\mybox}
  }

\usepackage{listings}
\lstset
{
    language=Caml,
    basicstyle=\footnotesize,
    numbers=left,
    stepnumber=1,
    showstringspaces=false,
    tabsize=1,
    breaklines=true,
    breakatwhitespace=false,
}

\setlist[itemize]{topsep=0pt}

\usepackage{hyperref}
\setlength{\parskip}{0.15cm}
\hypersetup{
    colorlinks=truem,
    linkcolor=black,
    filecolor=red,
    urlcolor=blue
}

\urlstyle{same}

\addbibresource{bib.bib}


\everymath{\displaystyle}
\title{\letitle}
\author{\leauthor}
\date{}
\begin{document}
\maketitle

\newpage

\section{Introduction}

Comme vu en théorie de la programmation, on peut faire une sémantique
dénotationnelle d'un langage de programmation.

C'est à dire envoyer les types du langage vers des ensembles.

En réalité on peut faire une telle sémantique dénotationnelle pas que pour les
ensembles mais pour toutes les catégories cartésiennes fermées.

La question est alors de savoir si on peut construire une telle sémantique vers
une catégorie des jeux et stratégies, que l'on va alors construire.

\section{Définitions}

Tentons de définir cette catégorie des jeux.

\begin{samepage}\textbf{\textit{\underline{ Structure d'événement (S.E.) : }}} \begin{answer}

Il s'agit de $E = (|E|, \leq_E, \#_E)$ où
\begin{itemize}
\item $\leq_E$ est un ordre partiel sur E
\item $\#_E$ est une relation binaire symétrique irréflexive :
\begin{itemize}
\item à causes finies :
$\forall e \in |E|, \{e' \in |E| \mid e' \leq_E e\}$ est fini

\item qui vérifie l'axiome de vendetta :
$\forall e_1 \#_E e_2, (e_2 \leq_E e'_2 \implies e_1 \# e'_2)$
\end{itemize}
\end{itemize}
\end{answer}\end{samepage}

\begin{samepage}\textbf{\textit{\underline{ Configuration : }}} \begin{answer}

$x$ fini $\subset |E|$ est une configuration si
\begin{itemize}
\item x est fermé vers le bas
($\forall e \in x, \forall e' \in |E|, e' \leq_E e \implies e' \in x$)

\item x est sans conflit
($\forall e, e' \in x, \neg (e \#_E e')$)
\end{itemize}
\end{answer}\end{samepage}
On note $\mathcal{C}(E)$ l'ensemble des configurations de $E$.

\begin{samepage}\textbf{\textit{\underline{ Justification : }}} \begin{answer}
$x \in \mathcal{C}(E)$ justifie $e \in |E|$, noté $x \vdash e$ si $x \uplus e \in \mathcal{C}(E)$
\end{answer}\end{samepage}

\begin{samepage}\textbf{\textit{\underline{ Jeu : }}} \begin{answer}
Un jeu est $A$ une S.E munie de $\text{pol}_A : |A| \rightarrow \{-, +\}$
\end{answer}\end{samepage}

\begin{samepage}\textbf{\textit{\underline{ Partie : }}} \begin{answer}
Une partie sur $A$ est un mot $s = s_1 \dots s_n \in |A|^*$ qui est :
\begin{itemize}
\item valide :
$\forall 1\leq i \leq n, \{s_1, \dots s_i\} \in \mathcal{C}(A)$

\item non répétitive :
$\forall i, j, s_i = s_j \implies i = j$

\item alternante :
$\forall 1 \leq i \leq n-1, \text{pol}_A(s_i) = - \text{pol}_A(s_{i+1})$

\item négative :
$s \neq \epsilon \implies \text{pol}_A(s_1) = -$
\end{itemize}
\end{answer}\end{samepage}
On note $\mathcal{P}(A)$ l'ensemble des parties sur $A$.

\subsection{Opérations sur les jeux}

\begin{samepage}\textbf{\textit{\underline{ Tenseur ou jeux parallèles : }}} \begin{answer}
On définit le jeu $A_1 \otimes A_2$
\begin{itemize}
\item $|A_1 \otimes A_2| = \{1\} \times |A_1| \cup \{2\} \times |A_2|$
\item $(i,a) \leq_{A_1 \otimes A_2} (j, a') \iff i = j $ et $a \leq_{A_i} a'$
\item $(i,a) \#_{A_1 \otimes A_2} (j,a') \iff i = j$ et $a \#_{A_i} a'$
\item $\text{pol}_{A_1 \otimes A_2}(i,a) = pol_{A_i} (a)$
\end{itemize}
\end{answer}\end{samepage}

\begin{samepage}\textbf{\textit{\underline{ Jeu dual : }}} \begin{answer}
On définit le jeu ${}^\bot A$
\begin{itemize}
\item $|{}^\bot A| = |A|$
\item $\leq_{{}^\bot A} = \leq_A$
\item $\#_{{}^\bot A} = \#_A$
\item $\text{pol}_{{}^\bot A} = - \text{pol}_A$
\end{itemize}
\end{answer}\end{samepage}

\begin{samepage}\textbf{\textit{\underline{ $A \vdash B$ : }}} \begin{answer}
$A \vdash B = {}^\bot A \otimes B$
\end{answer}\end{samepage}

\subsection{Stratégies sur les jeux}

\begin{samepage}\textbf{\textit{\underline{ Stratégie : }}} \begin{answer}
Une stratégie sur le jeu $A$ est $\sigma \subset \mathcal{P}(A)$ qui est
\begin{itemize}
\item non vide :
$\epsilon \in \sigma$

\item clos par préfixe :
$\forall e \in \sigma, \forall e', e' \sqsubseteq e \implies e' \in \sigma$

\item déterministe :
(Dans une partie quand c'est le tour du joueur +, il ne fait qu'un choix de
coup)\\
$sa_1^+, sa_2^+ \in \sigma \implies a_1 = a_2$

\item réceptive :
(Maximale dans un certains sens)\\
$s \in \sigma$ et $sa^- \in \mathcal{P}(A) \implies sa^- \in \sigma$
\end{itemize}
\end{answer}\end{samepage}

\begin{samepage}\textbf{\textit{\underline{ Restriction de stratégie : }}} \begin{answer}
Soit $s$ une stratégie sur $A_1 \otimes A_2$, alors $s\upharpoonright A_1$ est la
restriction de $s$ à $A_1$.

$s\upharpoonright A_1 = \{\Pi_{i \mid \exists b_i : a_i = (1, b_i)} a_i \mid a_1 \dots a_n \in s\}$

De même,

$s\upharpoonright A_2 = \{\Pi_{i \mid \exists b_i : a_i = (2, b_i)} a_i \mid a_1 \dots a_n \in s\}$
\end{answer}\end{samepage}

\subsection{Catégories :}

\begin{samepage}\textbf{\textit{\underline{ Catégorie : }}} \begin{answer}
Une catégorie $\mathcal{C}$ c'est
\begin{itemize}
\item Une classe d'objets $|\mathcal{C}|$

\item Pour chaque paire d'objets $A, B \in |\mathcal{C}|$, une classe de
morphismes $\mathcal{C}(A,B)$

\item Pour chaque $A \in |\mathcal{C}|$, un morphisme
$\text{id}_A \in \mathcal{C}(A,A)$

\item Une loi binaire sur les morphismes
$f \in \mathcal{C}(A,B)$, si $g \in \mathcal{C}(B,C)$, alors
$g \circ f \in \mathcal{C}(A,C)$
\end{itemize}
Tel que
\begin{itemize}
\item $\circ$ est associative:
$h \circ (g \circ f) = (h \circ g) \circ f$

\item Pour tout A, $\text{id}_A$ est neutre pour $\circ$ :
$f \circ \text{id}_A = \text{id}_A \circ f = f$
\end{itemize}

\end{answer}\end{samepage}

\section{La catégorie des jeux et des stratégies}

Nous allons tenter de construire la catégorie des jeux et stratégies

\begin{itemize}
\item Les objets seront les jeux
\item Les morphismes $\mathcal{C}(A,B)$ seront les stratégies sur le jeu $A \vdash B$
\item Pour A un objet, $\text{id}_A$ sera copycat de $A$ (défini très bientôt)
\item La composition de stratégies est définie très bientôt
\end{itemize}

Après avoir défini $\text{id}_A$ et $\circ$, il nous restera à prouver
que avec ces définitions, $\circ$ est associative et $\text{id}_A$ est neutre
pour $\circ$.

\subsection{Copycat}

Copycat de $A$ est une stratégie sur $A \vdash A = {}^\bot A \otimes A$

\begin{samepage}\textbf{\textit{\underline{ Definition : }}} \begin{answer}
Copycat de $A$ c'est
\begin{align*}
&cc_A\\
&=\{s \in \mathcal{P}(A_1 \vdash A_2) \mid \forall t=t_1 \dots t_n \sqsubseteq s/ pol_A(t_n) = + \implies t \upharpoonright A_1 = t \upharpoonright A_2\}
\end{align*}
\end{answer}\end{samepage}


\subsection{Composition de stratégies}

Soit $\sigma$ stratégie sur $A\vdash B$ et $\tau$ stratégie sur $B \vdash C$,
construisons $\tau \circ \sigma$ stratégie sur $A \vdash C$

\begin{samepage}\textbf{\textit{\underline{ Définition : interaction }}} \begin{answer}

Une interaction sur $A, B, C$ est $u \in (A \otimes B \otimes C)^*$ tel que
\begin{itemize}
\item $u \upharpoonright (A, B) \in \mathcal{P}(A \vdash B)$
\item $u \upharpoonright (B, C) \in \mathcal{P}(B \vdash C)$
\item $u \upharpoonright (A, C) \in \mathcal{P}(A \vdash C)$
\end{itemize}
\end{answer}\end{samepage}
On note $I(A,B,C)$ l'ensemble des intéractions sur $A,B,C$.

\begin{samepage}\textbf{\textit{\underline{ Définition : stratégies parallèles }}} \begin{answer}
$\sigma || \tau = \{u \in I(A,B,C) \mid u \upharpoonright A,B \in \sigma \text{ et }u \upharpoonright B,C \in \tau\}$
\end{answer}\end{samepage}

\begin{samepage}\textbf{\textit{\underline{ Définition : Composition de stratégies }}} \begin{answer}
$\sigma \circ \tau = \{u \upharpoonright A,C \mid u \in \sigma || \tau\}$
\end{answer}\end{samepage}

\begin{samepage}\textbf{\textit{\underline{ Diagramme de polarité }}} \begin{answer}
On partitionne $\mathcal{P}(A \vdash B)$ par la parité des longueurs de
partie.
Ainsi on a un ensemble $O$ des parties où c'est au tour de l'opposant
de jouer et $P$ celui des parties où c'est au tour du joueur de jouer.
On obtient $\mathcal{P}(A \vdash B) / \sim$

Soit un jeu $A$, on peut construire l'automate $\mathcal{A}_A$ où les états
sont les parties et les transitions sont les coups munis de leur polarisation
qui par justification mênent à une autre partie.
On peut aussi partitionner cet automate par $\sim$ pour avoir $\mathcal{A}_A/\sim$

Alors on construit le diagramme des polarités qui est l'automate

$(\mathcal{A}_{A \vdash B}/\sim) \times (\mathcal{A}_{B \vdash C}/\sim) \times (\mathcal{A}_{A \vdash C}/\sim)$

L'état initial étant $OOO$.

Par exemple depuis l'état $OOO$ on ne peut jouer un coup $b^+$ où $b \in |B|$
car ce serait un coup du joueur pour $A \vdash B$ alors que c'est le tour de
l'opposant.
De même on ne peut jouer un coup $b^-$ à cause de $B \vdash C$.

On aboutit alors au diagramme suivant :
\tikzfig{diagpol}


\end{answer}\end{samepage}

\begin{samepage}\textbf{\textit{\underline{ Proposition : }}} \begin{answer}
$\sigma \circ \tau$ ainsi défini est bien une stratégie sur $A \vdash C$
\end{answer}\end{samepage}

\begin{samepage}\textbf{\textit{\underline{ Preuve : }}} \begin{answer}

\begin{samepage}\textbf{\textit{\underline{ Lemme d'unicité du témoins : }}} \begin{answer}
Si $s \in \sigma \circ \tau$ est de longueur paire,
$\exists ! u \in \sigma||\tau $ tel que $ s = u \upharpoonright A,C$

on l'appelle le témoins de s.

\end{answer}\end{samepage}

\begin{samepage}\textbf{\textit{\underline{ Preuve du Lemme : }}} \begin{answer}
\begin{samepage}\textbf{\textit{\underline{ Existence : }}} \begin{answer}
Par définition de $\sigma \circ \tau$
\end{answer}\end{samepage}
\begin{samepage}\textbf{\textit{\underline{ Unicité : }}} \begin{answer}
Soit $u,v \in \sigma || \tau$ tels que
$u \upharpoonright A,C = s$ et $v \upharpoonright A, C = s$.

Soit $\omega$ leur plus long préfixe commun
\begin{alignat*}{3}
&u = &\omega &&l_1 &u'\\
&v = &\omega &&l_2 &v'
\end{alignat*}
\begin{itemize}
\item Si $l_1$ et  $l_2 \in B$.

Comme $u$ et $v$ sont alternants pour $\text{pol}_B$,
on a $\text{pol}_B(b_1) = \text{pol}_B(b_2)$

Supposons que ça vaut $+$.

Par déterminisme de $\tau$, comme
$(w\upharpoonright B,C)b_1^+ \in \tau$ et
$(w\upharpoonright B,C)b_2^+ \in \tau$,
on a $b_1$ = $b_2$.

\item Sinon

Nous verrons par la suite que c'est impossible étant donné
le diagramme de polarité

\end{itemize}



\end{answer}\end{samepage}
\end{answer}\end{samepage}

\end{answer}\end{samepage}
\printbibliography
\end{document}
